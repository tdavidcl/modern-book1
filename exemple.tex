\documentclass{researchbook}

\usepackage{lipsum}

\usepackage{marginnote}

\usepackage{fancyhdr}
\pagestyle{fancy}

\usepackage[svgnames]{xcolor}


\usepackage[sf,bf]{titlesec}

\titleformat{\section}
  {\normalfont\bfseries\Large\color{DarkRed}}{\thesection.}{1em}{}

%\usepackage[Glenn]{fncychap}
%\ChNameVar{\fontsize{14}{16}\usefont{OT1}{phv}{m}{n}\selectfont} \ChNumVar{\fontsize{60}{62}\usefont{OT1}{ptm}{m}{n}\selectfont} \ChTitleVar{\Huge\bfseries\rm}, \ChRuleWidth{1pt}
%\renewcommand{\thechapter}{\Roman{chapter}}
\titleformat{\chapter}[display]
{\bfseries\Large}
  {\filleft
  \MakeUppercase{\chaptertitlename} 
  \Huge\thechapter}
  {4ex}
  {\titlerule
   \vspace{2ex}%
   \filright}
  [\vspace{2ex}%
\titlerule\minitoc
]
\titlespacing{\chapter}{0pt}{-10pt}{20pt}



\usepackage{xcolor}
\definecolor{linkcolor}{rgb}{0,0,0.6} % définition de la couleur des liens pdf
\usepackage[ pdftex,colorlinks=true,
pdfstartview=ajustementV,
linkcolor= linkcolor,
citecolor= linkcolor,
urlcolor= linkcolor,
hyperindex=true,
hyperfigures=false]
{hyperref}

\usepackage{minitoc}


\usepackage{titling}
\setlength{\droptitle}{-20\baselineskip} % Move the title up

\pretitle{\begin{center}\Huge\bfseries} % Article title formatting
\posttitle{\end{center}} % Article title closing formatting


\usepackage{amsthm}

\usepackage[framemethod=tikz]{mdframed}

\theoremstyle{plain}
\newtheorem{theorem}[subsubsection]{Theorem}
\newtheorem{proposition}[subsubsection]{Proposition}
\newtheorem{lemma}[subsubsection]{Lemma}
%\newtheorem{proof}[subsubsection]{Proof}

\theoremstyle{definition}
\newtheorem{definition}[subsubsection]{Definition}
\newtheorem{example}[subsubsection]{Example}
\newtheorem{exercise}[subsubsection]{Exercise}
\newtheorem{situation}[subsubsection]{Situation}

\theoremstyle{remark}
\newtheorem{remark}[subsubsection]{Remark}
\newtheorem{remarks}[subsubsection]{Remarks}


\newcommand\updatestylethm[1]{\surroundwithmdframed[outerlinewidth=0.4pt, innerlinewidth=0.4pt, middlelinewidth=1pt, middlelinecolor=white, bottomline=true,topline=false,rightline=false,roundcorner=3pt]{#1}}

\updatestylethm{theorem}
\updatestylethm{lemma}
\updatestylethm{definition}

\updatestylethm{proposition}





\title{Cca}


\begin{document}
\dominitoc% Initialization



\maketitle
\newpage

PTDR la recherche

\tableofcontents


\part{First part}
\chapter{First chapter}

\section{First section}

Testing how an equation looks like : 

\[
\forall k, \quad \frac{\partial f^{[\{N_\alpha\}]}_{k}}{\partial t} = \sum_\alpha\left\{\mathcal{H}_\alpha, f^{[\{N_\alpha\}]}_k\right\} + \sum_{\alpha\beta} \left\{\mathcal{C}_{\alpha\beta}, f^{[\{N_\alpha\}]}_{k-1}\right\}
\]
\lipsum

\footnote{\lipsum}

\section{Second section}\lipsum
\reversemarginpar\marginnote{
    This is another margin note but shifted 2cm \textit{up} the page, relative to the line in which it is typeset. It is also in the left-hand margin.
}[-2cm]
\section{Third section}\lipsum
\section{Fourth section}\lipsum

\chapter{Second chapter}

\section{First section}\lipsum
\section{Second section}\lipsum
\section{Third section}\lipsum
\section{Fourth section}\lipsum


\chapter{Third chapter}

\section{First section}\lipsum
\section{Second section}\lipsum
\section{Third section}\lipsum
\section{Fourth section}\lipsum


\chapter{Fourth chapter}

\section{First section}\lipsum
\section{Second section}\lipsum
\section{Third section}\lipsum
\section{Fourth section}\lipsum


\part{Second part}
\chapter{First chapter}

\section{First section}\lipsum
\section{Second section}\lipsum
\section{Third section}\lipsum
\section{Fourth section}\lipsum


\chapter{Second chapter}

\section{First section}\lipsum
\section{Second section}\lipsum
\section{Third section}\lipsum
\section{Fourth section}\lipsum


\chapter{Third chapter}

\section{First section}\lipsum
\section{Second section}\lipsum
\section{Third section}\lipsum
\section{Fourth section}\lipsum


\chapter{Fourth chapter}

\section{First section}\lipsum
\section{Second section}\lipsum
\section{Third section}\lipsum
\section{Fourth section}\lipsum


\part{Third part}
\chapter{First chapter}

\section{First section}\lipsum
\section{Second section}\lipsum
\section{Third section}\lipsum
\section{Fourth section}\lipsum


\chapter{Second chapter}

\section{First section}\lipsum
\section{Second section}\lipsum
\section{Third section}\lipsum
\section{Fourth section}\lipsum


\chapter{Third chapter}

\section{First section}\lipsum
\section{Second section}\lipsum
\section{Third section}\lipsum
\section{Fourth section}\lipsum


\chapter{Fourth chapter}

\section{First section}\lipsum
\section{Second section}\lipsum
\section{Third section}\lipsum
\section{Fourth section}\lipsum




\end{document}
